%\documentclass[kulak,handout]{kulakbeamer} % handout
\documentclass[kulak,outline]{kulakbeamer} % outline

\usepackage[dutch]{babel}
\usepackage[T1]{fontenc}
\usepackage{amsmath,amssymb,amsthm}
\usepackage{siunitx}
    \sisetup{output-decimal-marker = {,}}
\usepackage{chemfig}
	\setchemfig{atom sep=2em}
\usepackage[version=3]{mhchem}
\DeclareMathOperator{\Bgtan}{Bgtan}
\newcommand{\R}{\ensuremath{\mathbb{R}}}
\newcommand{\norm}[1]{\left\|#1\right\|}
\newcommand{\pd}[2]{\frac{\partial #1}{\partial #2}}
\newenvironment{roodmidden}
{\begin{center}\color{red}}
	{\end{center}}
\usepackage{cancel}

\usefonttheme[onlymath]{serif}			% math font with serifs, delete to make it sans-serif

\title[\LaTeX\ en Beamer]{De kunst van het tekstzetten}
\subtitle{\LaTeX\ en Beamer}
\author{Stijn Rebry}
\institute[Kulak]{Groep Wetenschap \& Technologie, Kulak}
\date{X0B53a -- Wetenschappelijk rekenen en schrijven}
\renewcommand{\outlineTitle}{Overzicht}

\begin{document}
	
	\begin{titleframe}
	\titlepage
\end{titleframe}

\begin{outlineframe}\frametitle{Overzicht}
	\tableofcontents[hideallsubsections]
\end{outlineframe}

\section{Introductie}

\begin{frame}[t]
\frametitle{Wat is LaTeX?}
\begin{itemize}
	\item Stabiel, platform-onafhankelijk, betrouwbaar
	\begin{itemize}
		\item Gebaseerd op \TeX\ (1977, Donald Knuth)
		\item Macro's met voorgedefinieerde lay-out
		\item Nog steeds in ontwikkeling en achterwaarts compatibel
	\end{itemize}
	\item Mark-up in plaats van WYSIWYG
	\begin{itemize}
		\item Auteur bepaalt de logische structuur
		\item \LaTeX\ bepaalt de lay-out
	\end{itemize}
	\item Voor- en nadelen\ldots
	\begin{itemize}
		\item Automatisch een consequente, professionele lay-out
		\item Mooie en eenvoudig te maken formules
		\item Onmogelijk om ongestructureerde documenten te maken
	\end{itemize}
\end{itemize}
\end{frame}

% -----------------------------------------------------------

\begin{frame}[t]
\frametitle{Waaruit bestaat een LaTeX-systeem?}
\begin{enumerate}
\item Een geschikte editor
\begin{itemize}
	\item Bron is een gewoon tekst-bestand (\texttt{.tex})
	\item Windows: \texttt{TeXStudio}
\end{itemize}
\item Een \LaTeX-distributie
\begin{itemize}
	\item Zet \texttt{.tex}-tekst om naar \texttt{.pdf}-bestand
	\item Windows, Linux: \texttt{TeX Live}
	\item Mac OS: \texttt{MacTeX}
\end{itemize}
\item Een flexibele viewer
\begin{itemize}
	\item Printen, presenteren, doorzoeken van het \texttt{.pdf}-bestand
	\item Windows: \texttt{SumatraPDF}
\end{itemize}
\end{enumerate}
\end{frame}


\section{Wetenschappelijke artikels schrijven}

\subsection{Basissyntax van een \LaTeX-document}

\begin{frame}[fragile,t]
\frametitle{Een minimaal document}
\begin{columns}
\column{.5\textwidth}
\begin{exampleblock}{Minimaal document}
\begin{verbatim}
\documentclass{article}

\begin{document}

Dag iedereen!

\end{document}
\end{verbatim}
\end{exampleblock}
\column{.5\textwidth}
\verb+\documentclass{}+\newline bepaalt de lay-out:
\begin{itemize}
\item \texttt{article}: korte tekst
\item \texttt{book}: een boek
\item \texttt{beamer}:  een presentatie
\item \ldots
\end{itemize}
\end{columns}
\end{frame}

% -----------------------------------------------------------

\begin{frame}[fragile,t]
\frametitle{Structuur van een document}
\begin{columns}
\column{.5\textwidth}
\begin{block}{Minimale syntax}
\begin{verbatim}
\documentclass{article}

% Preamble

\begin{document}

% Body

\end{document}
\end{verbatim}
\end{block}
\column{.5\textwidth}
\begin{itemize}
\item Preamble: Instellingen\newline
van \verb+\documentclass+\newline
tot \verb+\begin{document}+
\item Body: Inhoud\newline van \verb+\begin{document}+\newline tot \verb+\end{document}+
\end{itemize}
\end{columns}
\end{frame}

% -----------------------------------------------------------

\begin{frame}[fragile,t]
\frametitle{Structuur van commando's}
\begin{columns}
\column{.5\textwidth}
\begin{block}{Syntax van commando's}
\begin{verbatim}
\commando{arg}
\commando{arg1}{arg2}
\commando[opt]{arg}

\begin{environment}
...
\end{environment}

% Commentaar
\end{verbatim}
\end{block}
\column{.5\textwidth}
\begin{itemize}
\item Commando's beginnen met \verb+\+ backslash 
\item Argumenten tussen \verb+{}+ accolades
\item Optionele argumenten tussen \verb+[]+ vierkante haakjes
\item Omgevingen/environments\newline
tussen \verb+\begin{...}+ en \verb+\end{...}+
\item Commentaar achter een \verb+%+ procent-teken
\end{itemize}
\end{columns}
\end{frame}

% -----------------------------------------------------------

\begin{frame}[fragile,t]
\frametitle{De hoofding}
\begin{columns}
\column{.5\textwidth}
\begin{exampleblock}{Titel, auteur en datum}
\begin{verbatim}
\documentclass{article}
\title{Van start ...}
\author{Stijn Rebry}
\date{\today}

\begin{document}
\maketitle
Dag iedereen!
\end{document}
\end{verbatim}
\end{exampleblock}
\column{.5\textwidth}
\begin{itemize}
\item \verb+\title+, \verb+\author+, \verb+\date+\newline leggen informatie vast
\item \verb+\today+\newline de datum van vandaag
\item \verb+\maketitle+\newline maakt de hoofding
\end{itemize}
\end{columns}
\end{frame}

% -----------------------------------------------------------

\begin{frame}[fragile,t]
\frametitle{Extra Pakketten}
\begin{columns}
\column{.5\textwidth}
\begin{exampleblock}{Taal, hyperlinks en figuren}
\begin{verbatim}
\documentclass{article}

\usepackage[dutch]{babel}
\usepackage{hyperref}
\usepackage{graphicx}

\title{Van start ...}
\end{verbatim}
\end{exampleblock}
\column{.5\textwidth}
Belangrijke pakketten:
\begin{itemize}
%\item \texttt{inputenc}: vreemde symbolen
\item \texttt{babel}: woordafbreking
\item \texttt{hyperref}: hyperlinks
\item \texttt{graphics}: figuren
\end{itemize}
\end{columns}
\end{frame}

% -----------------------------------------------------------

\begin{frame}[fragile,t]
\frametitle{Vreemde symbolen}
\begin{columns}
\column{.5\textwidth}
\begin{exampleblock}{Speciale tekens}
\begin{verbatim}

\# \$ \% \& \{ \}

\^ e \^{} e \~ n \~{} n

- \(-\) -- 

\LaTeX \LaTeX\ \LaTeX

\end{verbatim}
\end{exampleblock}
\column{.5\textwidth}
\begin{itemize}
\item Speciale symbolen:\newline
\# \$ \% \& \{ \}\newline
backslash toevoegen
\item Accenten en tilde:\newline
\^ e \^{} e \~ n \~{} n\newline
vereisen leeg argument
\item Koppelteken -, minteken \(-\), gedachte\-streepje --
\item Witruimte na commando verdwijnt, backslash + spatie zet expliciete spatie:\newline
\LaTeX \LaTeX\ \LaTeX
\end{itemize}
\end{columns}
\end{frame}


% -----------------------------------------------------------

\subsection{Structureren van een tekst}

\begin{frame}[fragile,t]
\frametitle{Hoofdstukken}
\begin{columns}
\column{.5\textwidth}
\begin{exampleblock}{(sub)\(^{\ast}\)secties}
\begin{verbatim}
\section*{Inleiding}
Ongenummerde sectie.

\section{Het begin}
Automatische nummering

\subsection{Subsectie}
Dubbele nummering 
\end{verbatim}
\end{exampleblock}
\column{.5\textwidth}
Verschillende niveau's
\begin{itemize}
\item \verb+\part+
\item \verb+\chapter+
\item \verb+\section+
\item \verb+\subsection+
\item \verb+\subsubsection+
\item \verb+\paragraph+
\item \verb+\subparagraph+
\end{itemize}
Inhoudstafel maken:
\begin{itemize}
\item \verb+\tableofcontents+
\end{itemize}
\end{columns}
\end{frame}

% -----------------------------------------------------------

\begin{frame}[fragile,t]
\frametitle{Opsommingen}
\begin{columns}
\column{.5\textwidth}
\begin{exampleblock}{Van begin tot end}
\begin{verbatim}
\begin{enumerate}

\item Editor

\item Distributie

\item Viewer

\end{enumerate}
\end{verbatim}
\end{exampleblock}
\column{.5\textwidth}
Soorten lijsten:
\begin{itemize}
\item \texttt{enumerate}
\item \texttt{itemize}
\item \texttt{description}
\end{itemize}
Nesten van lijsten mogelijk
\end{columns}
\end{frame}

% -----------------------------------------------------------

\begin{frame}[fragile,t]
\frametitle{Uitlijning}
\begin{columns}
\column{.5\textwidth}
\begin{exampleblock}{Van links naar rechts}
\begin{verbatim}
\begin{flushright}
\today
\end{flushright}

\begin{center}
In het midden
\end{center}
\end{verbatim}
\end{exampleblock}
\column{.5\textwidth}
Soorten uitlijning:
\begin{itemize}
\item \texttt{flushleft}
\item \texttt{flushright}
\item \texttt{center}
\item \texttt{quote}
\item \ldots
\end{itemize}
\end{columns}
\end{frame}

% -----------------------------------------------------------

\begin{frame}[fragile,t]
\frametitle{Computercode en typewriter-font}
\begin{columns}
\column{.5\textwidth}
\begin{exampleblock}{Computercode}
\verb|\texttt{typewriter}|\newline

\verb|\verb+\texttt{...}+|\newline
\verb=\verb+\verb|...|+=\newline

\verb|\begin{verbatim}|\newline
\verb| test := proc(x)|\newline
\verb|  x := x**2;|\newline
\verb| end proc:|\newline
\verb|\end{verbatim}|
\end{exampleblock}
\column{.5\textwidth}
Computercode steeds in \texttt{typewriter}-lettertype
\begin{itemize}
\item Doorlopende tekst:\newline
\verb|\texttt{...}|
\item LaTeX-code weergeven:\newline
\verb+\verb|...|+\newline
met willekeurig begin- en eindsymbool (hier \verb+|+)
\item Meerdere lijnen:\newline
\texttt{verbatim}-environment
\item Lange stukken code:\newline
\texttt{listings}-pakket
\end{itemize}
\end{columns}
\end{frame}

% -----------------------------------------------------------

\begin{frame}[fragile,t]
\frametitle{Kruisverwijzingen}
\begin{columns}
\column{.5\textwidth}
\begin{exampleblock}{Labels en referenties}
\begin{verbatim}
\section*{Inleiding}
In hoofdstuk \ref{sec:install}
op pagina \pageref{sec:install}
wordt eerst...

\section{Installatie}
\label{sec:install}
\subsection{MikTeX}
\label{sub:miktex} 
\end{verbatim}
\end{exampleblock}
\column{.5\textwidth}
\begin{itemize}
\item \verb+\label+ maakt een onzichtbaar etiket
\item \verb+\ref+ plaatst het correcte nummer
\item \verb+\pageref+ verwijst naar de pagina
\item 2 \(\times\) \LaTeX\ uitvoeren
\end{itemize}
\end{columns}
\end{frame}

% -----------------------------------------------------------

\begin{frame}[fragile,t]
\frametitle{Voetnoten}
Voetnoten\footnote{Dit is een voetnoot} in doorlopende tekst worden geplaatst met het commando \verb|\footnote|.
\begin{columns}
\column{.5\textwidth}
\begin{exampleblock}{Labels en referenties}
\begin{verbatim}
\footnote
{Dit is een voetnoot}

\footnotemark
\footnotetext
{Nog een voetnoot}
\end{verbatim}
\end{exampleblock}
\column{.5\textwidth}
\begin{itemize}
\item Doorlopende tekst:\newline
\verb|\footnote{...}|
\item Speciale omgevingen\footnotemark:\newline
\verb|\footnotemark|\newline
Verderop in tekst\newline
\verb|\footnotetext{...}|\newline
\end{itemize}
\end{columns}
\footnotetext{Nog een voetnoot\newline\newline ~}
\end{frame}

% -----------------------------------------------------------

\begin{frame}[fragile,t]
	\frametitle{Hyperlinks}
	\begin{columns}
		\column{.5\textwidth}
		\begin{exampleblock}{Hyperlinks}
			\begin{verbatim}
				\href{https://...}{text}
				
				\href{mailto:...}{text}
			\end{verbatim}
		\end{exampleblock}
		\column{.5\textwidth}
\verb+\href{URL}{tekst}+
		\begin{itemize}
			\item \texttt{URL}, de effectieve link
			\item \texttt{text}, aanklikbare tekst
			\item speciale symbolen worden aanvaard
			\item \texttt{mailto:...} voor e-mailadressen
		\end{itemize}
	\end{columns}
\end{frame}

% -----------------------------------------------------------

\subsection{Tabellen en figuren invoegen}

\begin{frame}[fragile,t]
\frametitle{Tabellen invoegen}
\begin{columns}
\column{.5\textwidth}
\begin{exampleblock}{Tabel maken}
\begin{verbatim}

\begin{tabular}{l|rc}
& kop 1 & kop 2 \\
\hline
rij1 &   abc &  123  \\
rij2 &     A &   1
\end{tabular}

\end{verbatim}
\end{exampleblock}
\column{.5\textwidth}
\begin{itemize}
\item \texttt{tabular}-omgeving
\item Kolomdefinitie \verb|{rcl}|: één symbool per kolom\newline
- \verb|r| rechts uitgelijnd\newline
- \verb|c| gecentreerd\newline
- \verb|l| links uitgelijnd\newline
- \verb+|+ vertikale lijn\newline
- \verb|\hline| horizontale lijn
\item \verb|&| kolomscheiding
\item \verb|\\| nieuwe rij
\end{itemize}
\end{columns}
\end{frame}

% -----------------------------------------------------------

\begin{frame}[fragile,t]
\frametitle{Tabellen invoegen}
\begin{columns}
\column{.5\textwidth}
\begin{exampleblock}{Speciale kolomdefinities}
\begin{verbatim}
\begin{tabular}{p{5cm}}
Tekst in een kolom van 5 cm breed
over meerdere lijnen
\end{tabular}

\begin{tabular}
	{r@{$\leftrightarrow$}l}
groot & klein\\ dik  & dun
\end{tabular}
\end{verbatim}
\end{exampleblock}
\column{.5\textwidth}
\begin{itemize}
\item Kolomdefinitie \verb|p{5cm}|:\newline
- kolom met vaste breedte\newline
- tekst over meerdere lijnen
\item Scheidingsteken tussen twee kolommen:  \verb|@{...}|
\item Tekst gecentreerd over twee kolommen:\newline
\verb|\multicolumn|\newline
\verb|   {2}{c}{text}|\newline
\end{itemize}
\end{columns}
\end{frame}

% -----------------------------------------------------------

\begin{frame}[fragile,t]
\frametitle{Figuur invoegen}
\begin{columns}
\column{.5\textwidth}
\begin{exampleblock}{Figuur invoegen}
\begin{verbatim}
\documentclass{article}
\usepackage{graphicx}
\begin{document}

\includegraphics
[width=.5\textwidth]
{figuur}

\end{document}
\end{verbatim}
\end{exampleblock}
\column{.5\textwidth}
\begin{itemize}
\item Pakket \texttt{graphic}\alert<2>{\texttt{x}} vereist
\item \verb+\includegraphic+\alert<2>{\texttt{s}}\texttt{\{bestand\}}\newline
figuur \texttt{bestand.???} invoegen
\item Extensie \texttt{.pdf}, \texttt{.png} of \texttt{.jpg},\newline
\alert<2>{niet \texttt{.jpeg}, \texttt{.JPG}, \texttt{.bmp}, \ldots}
\item Best in dezelfde map
\item Hoofdlettergevoelig
\item \alert<2>{Nooit} spaties in bestandsnamen
\item Optioneel argument:\newline grootte, rotatie, \ldots
\end{itemize}
\end{columns}
\end{frame}

% -----------------------------------------------------------

%\begin{frame}[fragile,t]
%\frametitle{Wetenschappelijke afbeeldingen}
%\begin{itemize}
%\item Rasterafbeeldingen (\texttt{.jpg} en \texttt{.png}) \alert{ongeschikt} voor grafieken
%%\item \texttt{.pdf}-export in Maple \alert{ongeschikt} (volledige pagina)
%%\item \texttt{.eps}-bestanden na conversie wel bruikbaar
%\item Klik in TeXStudio op \texttt{Options}, \texttt{Configure TeXstudio}, \texttt{Commands}, \texttt{PdfLaTeX}
%\item Voeg de optie \texttt{--shell-escape} toe:\newline
%\verb|pdflatex --shell-escape %.tex|
%\item Package \texttt{epstopdf} inladen
%\item \texttt{.eps}-figuur inladen zoals normaal\newline
%\verb|\includegraphics{figuur}|
%\end{itemize}
%\end{frame}

% -----------------------------------------------------------

\begin{frame}[fragile,t]
\frametitle{Float maken van tabel of figuur}
\begin{columns}
\column{.5\textwidth}
\begin{exampleblock}{Figuren laten zweven}
\begin{verbatim}
\begin{figure}
\centering
\includegraphics{...}
\caption{bijschrift}
\label{fig:euclides}
\end{figure}

\listoffigures
\end{verbatim}
\end{exampleblock}
\column{.5\textwidth}
\begin{itemize}
\item Zwevende omgeving met\newline
\texttt{figure} of \texttt{table}\newline
voor figuur resp.\ tabel
\item \verb+\centering+\newline inhoud centreren
\item \verb+\caption+ + \verb+\label+\newline
bijschrift voor kruisverwijzing
\item \verb|\listoffigures| en \verb|\listoftables| voor lijst van figuren resp.\ tabellen
\end{itemize}
\end{columns}
\end{frame}

% -----------------------------------------------------------

\subsection{Bibliografie en index}

\begin{frame}[fragile,t]
\frametitle{Bibliografische informatie verzamelen}
\begin{columns}
\column{.5\textwidth}
\begin{exampleblock}{Het \texttt{.bib}-bestand}
\begin{verbatim}
@article{label_article,
  title={...},
  author={...}}
@book{label_book,
  title={...},
  author={...}}
@misc{label_website,
  title={...},
  author={...}}
\end{verbatim}
\end{exampleblock}
\column{.5\textwidth}
\begin{itemize}
\item Afzonderlijk bestand met\newline bibliografische informatie\newline
\verb+mijnbib.bib+
\item Een record per bron,\newline
\verb+@article{...}+, \verb+@book{...}+, \verb+@misc{...}+, \ldots
\item Noodzakelijke informatie in velden naargelang type bron,\newline
\verb+title={...}+, \verb+author={...}+, \verb+publisher={...}+, \ldots
\end{itemize}
\end{columns}
\end{frame}

% -----------------------------------------------------------

\begin{frame}[fragile,t]
\frametitle{Bibliografische informatie invoegen}
\begin{columns}
\column{.5\textwidth}
\begin{exampleblock}{Citaties en bibliografie}
\begin{verbatim}
Uit het werk van Dierckx
\cite{label_article} blijkt dat...

Volgens de ECTS-fiche
\cite{label_website} geldt dat...

\bibliographystyle{plain}
\bibliography{mijnbib}
\end{verbatim}
\end{exampleblock}
\column{.5\textwidth}
\begin{itemize}
\item Citatie invoegen met behulp van het label,\newline
\verb+\cite{label}+
\item Integreer citaties in de tekst op een zinvolle plaats
\item Lay-out en ordening instellen met\newline
\verb+\bibliographystyle{...}+
\item Bibliografie invoegen met\newline
\verb+\bibliography{...}+
\item \LaTeX\ + Bib\TeX\ + \LaTeX
\end{itemize}
\end{columns}
\end{frame}

% -----------------------------------------------------------

\begin{frame}[fragile,t]
\frametitle{Index toevoegen}
\begin{columns}
\column{.5\textwidth}
\begin{exampleblock}{Het \texttt{makeidx}-pakket}
\begin{verbatim}
\usepackage{makeidx} \makeindex

\index{lemma}
\index{lemma!sublemma}
\index{bereik|(}...\index{bereik|)}
\index{lemma|see{...}}
\index{sorteren@lemma}

\printindex
\end{verbatim}
\end{exampleblock}
\column{.5\textwidth}
\begin{itemize}
\item Pakket inladen om \verb|\index{}|-commando beschikbaar te maken
\item \verb|\makeindex| zorgt dat de index effectief gemaakt wordt
\item \verb|\index{}| voegt lemma's toe, op de plaats naarwaar verwezen wordt
\item \verb|\printindex}| op de plaats waar de index wordt afgedrukt
\item \LaTeX\ + MakeIndex + \LaTeX
\end{itemize}
\end{columns}
\end{frame}

% -----------------------------------------------------------

\section{Wetenschappelijke formules invoegen}
\subsection{Wiskundige formules}

\begin{frame}[fragile,t]
\frametitle{De wiskunde-modi}
\begin{columns}
\column{.5\textwidth}
\begin{exampleblock}{Van klein naar groot}
\begin{verbatim}
Een rechte door punten \((a,b)\)
en \((c,d)\) heeft als vergelijking
\[y = \frac{d-a}{c-a}(x-a)+b.\]

Formule \ref{eq:euler} werd
ontdekt door Euler,
\begin{equation}\label{eq:euler}
	1+e^{i\pi}=0.
\end{equation}
\end{verbatim}
\end{exampleblock}
\column{.5\textwidth}
\begin{itemize}
\item \verb+$...$+ of \verb+\(...\)+\newline in doorlopende tekst
\item \verb+$$...$$+ of \verb+\[...\]+\newline op afzonderlijke lijn
\item \texttt{equation}-environment\newline genummerde formules\newline met label
\item \texttt{align}-environment\newline berekeningen over meerdere lijnen
\end{itemize}
\end{columns}
\end{frame}

% -----------------------------------------------------------

\begin{frame}[fragile,t]
\frametitle{De wiskunde-modi}
\begin{columns}
\column{.5\textwidth}
\begin{exampleblock}{Van klein naar groot}
\begin{verbatim}
\usepackage{amsmath}

\begin{align}
1 & = \sqrt{1}        \\
  & = \sqrt{(-1)(-1)} \\
  & = i \cdot i       \\
  & =  -1
\end{align}
\end{verbatim}
\end{exampleblock}
\column{.5\textwidth}
\begin{itemize}
\item \verb+$...$+ of \verb+\(...\)+\newline in doorlopende tekst
\item \verb+$$...$$+ of \verb+\[...\]+\newline op afzonderlijke lijn
\item \texttt{equation}-environment\newline genummerde formules\newline met label
\item \texttt{align}-environment\newline berekeningen over meerdere lijnen
\end{itemize}
\end{columns}
\end{frame}

% -----------------------------------------------------------

\begin{frame}[fragile,t]
\frametitle{Getallen}
    \begin{columns}
        \column{.5\textwidth}
\begin{exampleblock}{Commando \texttt{num} uit het \texttt{siunitx}-pakket}
\begin{tabular}{@{}r|r|r|r@{}}
Input&Text & Math & \verb|\num{...}| \\   \hline
\verb|1|&1          & $1      $       &\num{1       }     \\
\verb|-1|&\xcancel{-1}         & $-1     $       &\num{-1      }     \\
\verb|1,1|&1,1        & \xcancel{$1,1    $}       &\num{1,1     }     \\
\verb|10000|&\xcancel{10000}       & \xcancel{$10000   $}       &\num{10000    }     \\
\verb|0,00001|&\xcancel{0,00001}    & \xcancel{$0,00001$}       &\num{0,00001 }   
\end{tabular}
\begin{verbatim}
\usepackage{siunitx}
\sisetup{output-decimal-marker={,}}
\num{-123456.789}
\end{verbatim}

\end{exampleblock}
        \column{.5\textwidth}
Moeilijkheden:
\begin{itemize}
	\item Minteken
	\item Decimaalteken
	\item Spatiëring 
\end{itemize}
Oplossing naargelang soort getal:
\begin{itemize}
\item Natuurlijke getallen: geen probleem
\item Negatieve getallen: wiskunde-modus
\item Zeer grote getallen: \verb+num{...}+
\item Kommagetallen:  \verb+num{...}+
\end{itemize}
Enkel met \verb|\num{...}| altijd correct
    \end{columns}
\end{frame}

% -----------------------------------------------------------

\begin{frame}[fragile,t]
\frametitle{Roman of Italic?}
\begin{columns}
\column{.5\textwidth}
\begin{exampleblock}{Uitzonderingen}
\begin{verbatim}
\usepackage{amsmath}
\DeclareMathOperator{\Bgtan}{Bgtan}
\usepackage{siunitx,mhchem}

\cos(x), \Bgtan(x)
\(m_{\text{aarde}}\)
\(\si{kg.m/s^2}\)
\(\SI{9.81}{m/s^2}\)
\(V_{\ce{H_2O}}\)
\end{verbatim}
\end{exampleblock}
\column{.5\textwidth}
Wiskunde cursief tenzij
\begin{itemize}
\item Meerdere letters\\
niet \(cos x\) maar \(\cos x\)\\
niet \(Bgtan x\) maar \(\Bgtan x\)
\item Lopende tekst in formule\\
niet \(m_{aarde}\) maar \(m_{\text{aarde}}\)
\item Fysische eenheden\\
niet \(kg m/s^2\) maar \(\si{kg m/s^2}\)
\item Namen van atomen\\
niet \(V_{H_2O}\) maar \(V_{\ce{H2O}}\)
\end{itemize}
\end{columns}
\end{frame}

% -----------------------------------------------------------

\begin{frame}[fragile,t]
\frametitle{Functies en formules}
\begin{columns}
\column{.5\textwidth}
\begin{exampleblock}{Wiskunde-syntax}
\begin{verbatim}
x^n=\underbrace{
	x\cdot\ldots\cdot x}_n

\lim_{x\to 0}\frac{\sin(x)}{x}=1

\sum_{n=0}^{+\infty}\frac{1}{2^n}=2

(\int_a^x f(t) dt)'=f(x)
\end{verbatim}
\end{exampleblock}
\column{.5\textwidth}
\begin{itemize}
\item \verb+_+ subscript
\item \verb+^+ superscript
\item \verb+\frac+ breuk
\item \verb+\lim+ limiet
\item \verb+\sum+ som
\item \verb+\int+ integraal
\item \verb+\infty+ oneindig
\end{itemize}
\end{columns}
\end{frame}

% -----------------------------------------------------------

\begin{frame}[fragile,t]
\frametitle{Matrices en stelsels}
\begin{columns}
\column{.5\textwidth}
\begin{exampleblock}{Syntax voor  matrices}
\begin{verbatim}
\left( \begin{array}{cc}
   a_{11} & a_{12} \\
   a_{21} & a_{22} 
\end{array} \right)

|x| =  \left\{ \begin{array}{rl}
    x & \textrm{als } x   >  0 \\
   -x & \textrm{als } x \leq 0 
\end{array} \right.
\end{verbatim}
\end{exampleblock}
\column{.5\textwidth}
\begin{itemize}
\item \texttt{array}-omgeving, analoog aan \texttt{tabular}
%\item Kolomdefinitie \verb|{rcl}|: één symbool per kolom\newline
%- \verb|r| rechts uitgelijnd\newline
%- \verb|c| gecentreerd\newline
%- \verb|l| links uitgelijnd\newline
%- \verb+|+ vertikale streep\newline
%- \verb|\hline| horizontale streep
%\item \verb|&| kolomscheiding
%\item \verb|\\| nieuwe rij
\item \verb+\left(+ en \verb+\right)+ gepaste haakjes
\item \verb+\left\{+ en \verb+\right.+ voor stelsel
\end{itemize}
Tabellen allerhande:
\begin{itemize}
	\item \texttt{tabular}: teksmodus, elementen horen tekst te zijn, of \verb+\(\)+
	\item \texttt{array}: wiskundemodus, elementen horen wiskunde te zijn, of \verb+\text{...}+
	\item \texttt{table}: float-environment voor tabellen, bevat typisch nog \texttt{tabular}-environment
\end{itemize}
\end{columns}
\end{frame}

% -----------------------------------------------------------

\subsection{Eigen commando's en omgevingen}

\begin{frame}[fragile,t]
\frametitle{Eigen commando's definiëren}
\begin{columns}
\column{.5\textwidth}
\begin{exampleblock}{Nieuwe commando's}
\begin{verbatim}
\usepackage{amssymb}
\newcommand{\R}
   {\ensuremath{\mathbb{R}}}
\newcommand{\norm}[1]
   {\left\|#1\right\|}
\newcommand{\pd}[2]
   {\frac{\partial #1}{\partial #2}}

\(\R, \norm{\vec{v}, \pd{f}{x}\)
\end{verbatim}
\end{exampleblock}
\column{.5\textwidth}
\verb+\newcommand{\naam}[n]{<code>}+
\begin{itemize}
\item \verb+\naam+  het nieuwe commando
\item \verb+n+ aantal argumenten
\item \verb|<code>| uit te voeren code
\item \verb|#1|, \verb|#2| argumenten
\end{itemize}
Resultaten:
\begin{itemize}
\item \R \\
\item \(\norm{\vec{v}}\) \\
\item \(\pd{f}{x}\)
\end{itemize}
\end{columns}
\end{frame}

% -----------------------------------------------------------

\begin{frame}[fragile,t]
\frametitle{Eigen omgevingen definiëren}
\begin{columns}
\column{.5\textwidth}
\begin{exampleblock}{Definities en stellingen}
\begin{verbatim}
\newenvironment{roodmidden}
   {\begin{center} \color{red}}
   {\end{center}}

\begin{roodmidden}
   Rood en in het midden
\end{roodmidden}
\end{verbatim}
\end{exampleblock}
\column{.5\textwidth}
\verb+\newenvironment{naam}+\\
\verb+ {<begin>} {<einde>}+
\begin{itemize}
\item \verb+naam+ de nieuwe omgeving
\item \verb+\begin{naam}+ wordt \verb+<begin>+
\item \verb+\end{naam}+ wordt \verb+<einde>+
\end{itemize}
Resultaat:
\begin{roodmidden}
Rood en in het midden
\end{roodmidden}
\end{columns}
\end{frame}

% -----------------------------------------------------------

\begin{frame}[fragile,t]
\frametitle{Stelling en bewijs}
\begin{columns}
\column{.5\textwidth}
\begin{block}{newtheorem-syntax}
\begin{verbatim}
% nieuwe teller:
\newtheorem
{arg1}{arg2}[opt1]

% bestaande teller:
\newtheorem
{arg1}[opt2]{arg2}
\end{verbatim}
\end{block}
\column{.5\textwidth}
\begin{itemize}
\item \verb+\newtheorem+\newline nieuwe omgeving
\item \texttt{arg1}: naam environment
\item \texttt{arg2}: Stelling/Definitie
\item \texttt{opt1}: sectie-niveau
\item \texttt{opt2}: theorem-naam
\end{itemize}
\end{columns}
\end{frame}

% -----------------------------------------------------------

\begin{frame}[fragile,t]
\frametitle{Stelling en bewijs}
\begin{columns}
\column{.48\textwidth}
\begin{exampleblock}{Preambule}
\begin{verbatim}
	\newtheorem{df}{Definitie}[section]
	\newtheorem{st}[df]{Stelling}
\end{verbatim}
\end{exampleblock}
\column{.52\textwidth}
\begin{exampleblock}{Body}
\begin{verbatim}
	\begin{df}[Continuïteit]
		Een functie is continu als ...
	\end{df}
	\begin{st}
		De som van twee continue functies ...
	\end{st}
	\begin{proof}
		Oefening voor de lezer.
	\end{proof}
\end{verbatim}
\end{exampleblock}
\end{columns}
\end{frame}

% -----------------------------------------------------------

\subsection{Chemische moleculen en reacties}

\begin{frame}[fragile,t]
\frametitle{Basisformules}
\begin{columns}
\column{.5\textwidth}
\begin{exampleblock}{Chemie-modus}
\begin{verbatim}
\usepackage{mhchem}

\begin{document}

\ce{^{227}_{90}Th^+}
\ce{CO_2 + C <=> 2CO}
\ce{SO_4^{2-}+Ba^{2+} -> BaSO_4 v}
\end{verbatim}
\end{exampleblock}
\column{.5\textwidth}
\begin{itemize}
\item Pakket: \texttt{mhchem}
\item Commando: \verb|\ce{...}|
\end{itemize}

Resultaten:
\begin{itemize}
\item \ce{^{227}_{90}Th^+}
\item \ce{CO_2 + C <=> 2CO}
\item \ce{SO_4^{2-} + Ba^{2+} -> BaSO_4 v}
\end{itemize}
\end{columns}
\end{frame}

% -----------------------------------------------------------

\begin{frame}[fragile,t]
\frametitle{Moleculen tekenen}
\begin{columns}
\column{.5\textwidth}
\begin{exampleblock}{Chemie-modus}
\begin{verbatim}
\usepackage{chemfig}
\setchemfig{atom sep=2em}

\chemfig{H-[:30]O-[:30]H}
\chemfig{-[:30]=[:-30]-[:30]}
\chemfig{*6(=-=-=-)}
\chemfig{N(-[:150]H)(-[:-150]H)
	(-C*6(=C-C=C-C=C-C))}
\end{verbatim}
\end{exampleblock}
\column{.5\textwidth}
\begin{itemize}
\item Pakket: \texttt{chemfig}
\item Commando: \verb|\chemfig{...}|
\end{itemize}
Resultaten:
\begin{center}
\chemfig{H-[:30]O-[:-30]H}
\chemfig{-[:30]=[:-30]-[:30]}
\chemfig{*6(=-=-=-)}
\chemfig{N(-[:150]H)(-[:-150]H)(-C*6(=C-C=C-C=C-))}
\end{center}
\end{columns}
\end{frame}


% -----------------------------------------------------------
\section{Presentaties maken met Beamer}

\subsection{De basisprincipes van Beamer}

\begin{frame}[fragile,t]
\frametitle{De \texttt{Beamer}-klasse}
\begin{columns}
\column{.5\textwidth}
\begin{exampleblock}{Een minimaal document}
\begin{verbatim}
\documentclass[...]{beamer}

\title[Kort]{Volledig}
\author[Kort]{Volledig}
\date{\today}

\end{verbatim}
%\verb|\documentclass|\\
%\verb| %[handout]|\\
%\verb| {beamer}|\\
%\verb| \title[Kort]{Volledig}|\\
%\verb| \author[Kort]{Volledig}|\\
%\verb| \date{\today}|\\
%\verb|\begin{document}|\\
%\verb| \begin{frame}|\\
%\verb|  \titlepage|\\
%\verb| \end{frame}|\\
%\verb|\end{document}|
%\usetheme{PaloAlto}
%\usecolortheme{seahorse}
%  
%\setbeamercovered
% {transparent}
%
%\logo
% {\includegraphics{...}}
\end{exampleblock}
\column{.5\textwidth}
\begin{itemize}
\item \verb|beamer|\newline
presentatie-opmaak
\item \verb+handout+ (optioneel)\newline enkel handouts maken
\item \verb+\titlepage+\newline maakt een titel-slide
%\item \verb|frame|:\newline
%bevat telkens één scherm-inhoud
\item Korte versie van titel, auteur, secties, \ldots voor kop- en voettekst
%\textit{\item \verb+\usetheme+:\newline  lay-out van de slides\newline
%\texttt{Madrid}, \texttt{Antibes}, \ldots
%\item \verb+\usetheme+:\newline  kleurenschema\newline
%\texttt{albatross}, \texttt{beetle}, \ldots
%\item \verb+\setbeamercovered+:\newline  transparante tekst
%\item \verb+\logo+:\newline  een logo invoegen}
\end{itemize}
\end{columns}
\end{frame}

% -----------------------------------------------------------

\begin{frame}[fragile,t]
\frametitle{Secties en frames}
\begin{columns}
\column{.5\textwidth}
\begin{exampleblock}{Een eerste frame}
\begin{verbatim}
\begin{document}

\section
{Basisprincipes}
\end{verbatim}
\verb|\begin{frame}|

\verb|\frametitle|
\verb|{Secties en frames}|
\verb|Inhoud van het frame|
\verb|\end{frame}|
\begin{verbatim}
\end{document}
\end{verbatim}
%\begin{verbatim}
%\begin{document}
%
%\section
%{Basisprincipes}
%
%\begin{frame}
%\frametitle
%{Secties en frames}
%Inhoud van het frame
%\end{frame}
%
%\end{document}
%\end{verbatim}
\end{exampleblock}
\column{.5\textwidth}
\begin{itemize}
\item Alle inhoud in \texttt{frame}-environment: telkens één scherminhoud
\item \verb+\frametitle+ geeft elk frame afzonderlijke titel
\item Sectie-commando's daarbuiten: bepalen inhoudstafel en overzicht
\end{itemize}
\end{columns}
\end{frame}

% -----------------------------------------------------------

\subsection[Structureren]{Frames structureren}

\begin{frame}[fragile,t]
\frametitle{Horizontaal structureren}
\begin{columns}
\column{.5\textwidth}
\begin{exampleblock}{Block-syntax}
\begin{verbatim}
\begin{block} {Een blok}
...
\end{block}
\begin{exampleblock} {Een voorbeeld}
...
\end{exampleblock}
\begin{alertblock} {Een opmerking}
...
\end{alertblock}
\end{verbatim}
\end{exampleblock}
\column{.5\textwidth}
\begin{itemize}
\item \texttt{block}:\newline belangrijk resultaat, doorgaans blauw
\item \texttt{exampleblock}:\newline een voorbeeld, groen
\item \texttt{alertblock}:\newline een opmerking, rood
\item \texttt{theorem}:\newline automatisch als \texttt{block}
\end{itemize}
\end{columns}
\end{frame}

% -----------------------------------------------------------

\begin{frame}[fragile,t]
\frametitle{Verticaal structureren}
\begin{columns}
\column{.5\textwidth}
\begin{exampleblock}{Columns-syntax}
\begin{verbatim}
\begin{columns}
\column{.5\textwidth}
Tekst in kolom 1
\column{.5\textwidth}
Tekst in kolom 2
\end{columns}
\end{verbatim}
\end{exampleblock}
\column{.5\textwidth}
\begin{itemize}
\item \texttt{columns}-environment:\newline deelt de beschikbare breedte op in meerdere kolommen
\item \verb+\column{}+:\newline begint een nieuwe kolom van de opgegeven breedte
\end{itemize}
\end{columns}
\end{frame}

% -----------------------------------------------------------

\subsection{Slides}

\begin{frame}[fragile,t]
\frametitle{Slides en frames}
\begin{columns}
\column{.5\textwidth}
\begin{exampleblock}{Pauzeren}
\begin{verbatim}
\begin{itemize}

\item Eerste slide

\pause

\item Tweede slide

\end{itemize}
\end{verbatim}
\end{exampleblock}
\column{.5\textwidth}
\begin{itemize}[<+->]
\item \texttt{Frame}: De volledige inhoud van één scherm
\item \texttt{Slide}: Elke stap binnen de opbouw van een frame
\item \verb+\pause+: Pauzeert tussen twee slides
\end{itemize}
\end{columns}
\end{frame}

% -----------------------------------------------------------

\begin{frame}[fragile,t]
\frametitle{Overlays}
\begin{columns}
\column{.5\textwidth}
\begin{exampleblock}{Speciale effecten}
\begin{verbatim}
\uncover<2-> {Leesbaar maken}

\only <3-4> {Laten verschijnen}

\alert <4> {Benadrukken}

\textit <5> {Tekstopmaak toepassen}
\end{verbatim}
\end{exampleblock}
\column{.5\textwidth}
\begin{itemize}
\item \verb+<2->+:\newline slide specificatie
\item \verb+\uncover+:\newline  \uncover<2->{Leesbaar maken}
\item<only@3-4> \verb+\only+:\newline  Tekst laten verschijnen
\item \verb+\alert+:\newline  \alert<4>{Tekst benadrukken}
\item \verb+\textit+:\newline  \textit<5>{Tekstopmaak toepassen}
\end{itemize}
\end{columns}
\end{frame}

% -----------------------------------------------------------

\begin{frame}[fragile,t]
\frametitle{Overlays in opsommingen}
\begin{columns}
\column{.5\textwidth}
\begin{exampleblock}{Slidespecificaties}
\begin{verbatim}
\item<2> Enkel op slide 2

\item<-4> Tot en met slide 4

\item<3-5> Van slide 3 tot 5

\item<2-> Vanaf slide 2

\item<-2|alert@4> Combinatie
\end{verbatim}
\end{exampleblock}
\column{.5\textwidth}
Slide \only<1>{1}\only<2>{2}\only<3>{3}\only<4>{4}\only<5>{5}
\begin{itemize}
\item<2> Enkel op slide 2
\item<-4> Tot en met slide 4
\item<3-5> Van slide 3 tot slide 5
\item<2-> Vanaf slide 2
\item<-2,4|alert@4> Tot slide twee en op slide 4 benadrukt
\end{itemize}
\end{columns}
\end{frame}

% -----------------------------------------------------------

\begin{frame}[fragile,t]
\frametitle{Overlays in opsommingen}
\begin{columns}
\column{.5\textwidth}
\begin{exampleblock}{Automatische specificaties}
\begin{verbatim}
\begin{itemize}[<+->]

\item Items automatisch

\item één voor één

\item laten verschijnen

\end{itemize}
\end{verbatim}
\end{exampleblock}
\column{.5\textwidth}
\begin{itemize}[<+->]

\item Items automatisch

\item één voor één

\item laten verschijnen

\end{itemize}

\end{columns}
\end{frame}

% -----------------------------------------------------------

\begin{frame}[fragile,t]
\frametitle{Kruisverwijzingen in beamer}
\begin{columns}
\column{.5\textwidth}
\begin{exampleblock}{Labels op specifieke slides}
\begin{verbatim}
\hypertarget <1> {label1} {tekst1}

\hypertarget <2> {label2} {tekst2}

\hyperlink {label1} {tekst}
\end{verbatim}
\end{exampleblock}
\column{.5\textwidth}
\begin{itemize}
\item \verb|\hypertarget<1>{label1}{tekst1}|
\begin{itemize}
\item \verb|<1>|: slide die label bevat
\item \verb|{label1}|: onzichtbaar label
\item \verb|{tekst1}|: bijhorende tekst
\end{itemize}
\item \verb|\hyperlink{label1}{tekst}|
\begin{itemize}
\item \verb|{label1}|: bestaand label
\item \verb|{tekst}|: aanklikbare tekst
\end{itemize}
\end{itemize}

\end{columns}
\end{frame}

% -----------------------------------------------------------

\subsection{Handouts maken}

\begin{frame}[fragile,t]
\frametitle{Handouts maken}
\begin{columns}
\column{.5\textwidth}
\begin{exampleblock}{Beamer vs.\ handout}
\begin{verbatim}
\documentclass[handout]{beamer}
\AtBeginSection{...}

\begin{document}

\only<beamer|handout>{...}

\begin{frame}[label=overzicht]
\againframe{overzicht}
\end{verbatim}
\end{exampleblock}
\column{.5\textwidth}
\begin{itemize}
\item Standaard: elke slide is één fysieke pagina
\item Optie \texttt{handout}: slechts één pagina per frame
\item Slide-specificaties \only<beamer>{\texttt{beamer}: enkel in de presentatie} \only<handout>{\texttt{handout}: enkel in de handout}
\item Nuttig om inhoudstafel of specifieke slides te hernemen
\end{itemize}
\end{columns}
\end{frame}

% -----------------------------------------------------------

\section{TeXStudio, templates en meer}

% -----------------------------------------------------------

\begin{frame}[fragile,t]
	\frametitle{Wetenschappelijke poster met \texttt{kulakposter}}
	\begin{columns}
		\column{.5\textwidth}
		\begin{exampleblock}{Kolommen op een poster}
			\begin{verbatim}
				\usepackage{multicol}
				
				\begin{multicols}{2}
					Kolom 1
					\columnbreak
					Kolom 2
				\end{multicols}
			\end{verbatim}
		\end{exampleblock}
		\column{.5\textwidth}
		\begin{itemize}
			\item Gebaseerd op \texttt{sciposter}-klasse.
			\item Kolommen met \texttt{multicol}-pakket:\newline
			- \texttt{multicols}-environment
			- kan genest worden\newline
			- kolommen automatisch gebalanceerd, manueel met \verb|\columnbreak|
			\item Zo weinig mogelijk doorlopende tekst, zo veel mogelijk afbeeldingen
			\item Floats zelf positioneren.
		\end{itemize}
	\end{columns}
\end{frame}


\subsection{KU Leuven en Kulak huisstijl}

\begin{frame}[fragile,t]
\frametitle{KU Leuven en Kulak huisstijl: bestanden}
Beschikbare documentklassen:
\begin{itemize}
	\item \texttt{kulakarticle}, eenzijdig, eenvoudige hoofding
	\item \texttt{kulakreport}, tweezijdig, voorblad en achterflap
	\item \texttt{kulakposter}, A3 tot A0, landscape of portrait-poster
	\item \texttt{kulakbeamer}, beamer- en handout-klasse
\end{itemize}
%Beschikbare beamersjablonen:
%\begin{itemize}
%	\item \texttt{beamerthemekuleuven2}
%\end{itemize}
%\only<2>{\texttt{kulakbeamer} is verouderd, gebruik het beamersjabloon voor nieuwe documenten!}
\end{frame}

\begin{frame}[fragile,t]
\frametitle{KU Leuven en Kulak huisstijl: installatie}
Zie \texttt{readme.pdf} (ook voor Mac OS X/Linux))
\begin{itemize}
\item Kopieer de \texttt{texmf}-submap \texttt{kulak} naar\newline
\verb|c:\texlive\texmf-local\tex\latex\local|
\item Start het programma \texttt{TeX Live Manager} via\newline
\texttt{Start}, \texttt{All programs}, \texttt{TeX Live yyyy}, \texttt{TeX Live Manager}
\item Klik op \texttt{Actions}, \texttt{Update filename database}
\end{itemize}
\end{frame}

% -----------------------------------------------------------

\begin{frame}[fragile,t]
\frametitle{KU Leuven en Kulak huisstijl: gebruik}
Zie \texttt{readme.pdf} en meegeleverde templates
\begin{itemize}
\item Voor documentklassen:
\begin{verbatim}
\documentclass[a4paper,kulak]{kulakarticle}
\end{verbatim}
\item Voor beamersjabloon:
\begin{verbatim}
\documentclass[aspectratio=169,kulak,t]{kulakbeamer}
\end{verbatim}
\end{itemize}
\end{frame}

% -----------------------------------------------------------

\subsection{Templates in TeXStudio}

\begin{frame}[fragile,t]
\frametitle{Templates instellen in TeXStudio}

Eigen template maken:
\begin{itemize}
\item Schrijf minimaal \texttt{.tex}-bestand
\item Klik op \texttt{File}, \texttt{Make Template...}
\end{itemize}

Kulak-templates installeren:
\begin{itemize}
\item Plaats inhoud van \texttt{Templates}-map in\newline
\verb|C:\users\<user>\AppData\Roaming\texstudio\templates\user|\newline
\verb|~/.config/texstudio/templates/user|
\end{itemize}

Template openen:
\begin{itemize}
\item Klik op \texttt{File}, \texttt{New From Template...}
\end{itemize}
\end{frame}

% -----------------------------------------------------------

\subsection{Spellingscorrectie in TeXStudio}

\begin{frame}[fragile,t]
\frametitle{Spellingscorrectie in TeXStudio}

Woordenlijst downloaden en installeren:
\begin{itemize}
\item Per gewenste taal OpenOffice-add on downloaden,\newline
\href{http://extensions.openoffice.org/dictionaries}{\texttt{http://extensions.openoffice.org/dictionaries}}
%\item Extensie \texttt{.oxt} veranderen in \texttt{.zip}
%\item \texttt{.dic}-bestand uit \texttt{.zip}-file kopiëren naar\newline
%\verb|C:/Program Files (x86)/TeXstudio/dictionaries|
\item Voorkeuren in TeXStudio instellen via
\texttt{Options}, \texttt{Configure TeXStudio}, \texttt{General}, \texttt{Dictionaries}
\item Taal per taal toevoegen via \texttt{Import dictionary...} en gewenste \texttt{.oxt}-bestand selecteren
\end{itemize}
Taal van document instellen
\begin{itemize}
\item Selecteer juiste taal rechts onderaan TeXStudio-scherm
\item Klik op \texttt{Insert Language as TeX comment}
\end{itemize}
\end{frame}

% -----------------------------------------------------------

\subsection{Meer informatie}

\begin{frame}
\frametitle{Meer informatie}
\begin{itemize}
\item\href{http://mirrors.ctan.org/info/lshort/english/lshort.pdf}{The not so Short Introduction to \LaTeX2e}:\newline
De beste introductie tot \LaTeX
\item\href{http://mirrors.ctan.org/macros/latex/contrib/beamer/doc/beameruserguide.pdf}{The \textsc{beamer} class}:
Uitgebreide handleiding bij \textsc{beamer},\newline
Sectie 3, \emph{Euclid’s Presentation}, is goede introductie
\item\href{http://mirrors.ctan.org/graphics/pgf/base/doc/pgfmanual.pdf}{The Ti\emph{k}Z \& \textsc{PGF} Packages}: Handleiding bij tekenpakket,\newline
Deel I, \emph{Tutorials and Guidelines}, is goede introductie
\item \href{http://www.ctan.org/}{The Comprehensive TeX Archive Network}:\newline
Uitgebreide databank met handleiding bij elk pakket.
\end{itemize}

\end{frame}


\end{document}
